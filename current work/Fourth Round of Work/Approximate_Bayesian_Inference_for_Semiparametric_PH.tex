\documentclass[]{article}
\usepackage{lmodern}
\usepackage{amssymb,amsmath}
\usepackage{ifxetex,ifluatex}
\usepackage{fixltx2e} % provides \textsubscript
\ifnum 0\ifxetex 1\fi\ifluatex 1\fi=0 % if pdftex
  \usepackage[T1]{fontenc}
  \usepackage[utf8]{inputenc}
\else % if luatex or xelatex
  \ifxetex
    \usepackage{mathspec}
  \else
    \usepackage{fontspec}
  \fi
  \defaultfontfeatures{Ligatures=TeX,Scale=MatchLowercase}
\fi
% use upquote if available, for straight quotes in verbatim environments
\IfFileExists{upquote.sty}{\usepackage{upquote}}{}
% use microtype if available
\IfFileExists{microtype.sty}{%
\usepackage{microtype}
\UseMicrotypeSet[protrusion]{basicmath} % disable protrusion for tt fonts
}{}
\usepackage[margin=1in]{geometry}
\usepackage{hyperref}
\hypersetup{unicode=true,
            pdftitle={Approximate Bayesian Inference for Semi-parametric Proportional Hazard Models},
            pdfauthor={Ziang Zhang},
            pdfborder={0 0 0},
            breaklinks=true}
\urlstyle{same}  % don't use monospace font for urls
\usepackage{color}
\usepackage{fancyvrb}
\newcommand{\VerbBar}{|}
\newcommand{\VERB}{\Verb[commandchars=\\\{\}]}
\DefineVerbatimEnvironment{Highlighting}{Verbatim}{commandchars=\\\{\}}
% Add ',fontsize=\small' for more characters per line
\usepackage{framed}
\definecolor{shadecolor}{RGB}{248,248,248}
\newenvironment{Shaded}{\begin{snugshade}}{\end{snugshade}}
\newcommand{\AlertTok}[1]{\textcolor[rgb]{0.94,0.16,0.16}{#1}}
\newcommand{\AnnotationTok}[1]{\textcolor[rgb]{0.56,0.35,0.01}{\textbf{\textit{#1}}}}
\newcommand{\AttributeTok}[1]{\textcolor[rgb]{0.77,0.63,0.00}{#1}}
\newcommand{\BaseNTok}[1]{\textcolor[rgb]{0.00,0.00,0.81}{#1}}
\newcommand{\BuiltInTok}[1]{#1}
\newcommand{\CharTok}[1]{\textcolor[rgb]{0.31,0.60,0.02}{#1}}
\newcommand{\CommentTok}[1]{\textcolor[rgb]{0.56,0.35,0.01}{\textit{#1}}}
\newcommand{\CommentVarTok}[1]{\textcolor[rgb]{0.56,0.35,0.01}{\textbf{\textit{#1}}}}
\newcommand{\ConstantTok}[1]{\textcolor[rgb]{0.00,0.00,0.00}{#1}}
\newcommand{\ControlFlowTok}[1]{\textcolor[rgb]{0.13,0.29,0.53}{\textbf{#1}}}
\newcommand{\DataTypeTok}[1]{\textcolor[rgb]{0.13,0.29,0.53}{#1}}
\newcommand{\DecValTok}[1]{\textcolor[rgb]{0.00,0.00,0.81}{#1}}
\newcommand{\DocumentationTok}[1]{\textcolor[rgb]{0.56,0.35,0.01}{\textbf{\textit{#1}}}}
\newcommand{\ErrorTok}[1]{\textcolor[rgb]{0.64,0.00,0.00}{\textbf{#1}}}
\newcommand{\ExtensionTok}[1]{#1}
\newcommand{\FloatTok}[1]{\textcolor[rgb]{0.00,0.00,0.81}{#1}}
\newcommand{\FunctionTok}[1]{\textcolor[rgb]{0.00,0.00,0.00}{#1}}
\newcommand{\ImportTok}[1]{#1}
\newcommand{\InformationTok}[1]{\textcolor[rgb]{0.56,0.35,0.01}{\textbf{\textit{#1}}}}
\newcommand{\KeywordTok}[1]{\textcolor[rgb]{0.13,0.29,0.53}{\textbf{#1}}}
\newcommand{\NormalTok}[1]{#1}
\newcommand{\OperatorTok}[1]{\textcolor[rgb]{0.81,0.36,0.00}{\textbf{#1}}}
\newcommand{\OtherTok}[1]{\textcolor[rgb]{0.56,0.35,0.01}{#1}}
\newcommand{\PreprocessorTok}[1]{\textcolor[rgb]{0.56,0.35,0.01}{\textit{#1}}}
\newcommand{\RegionMarkerTok}[1]{#1}
\newcommand{\SpecialCharTok}[1]{\textcolor[rgb]{0.00,0.00,0.00}{#1}}
\newcommand{\SpecialStringTok}[1]{\textcolor[rgb]{0.31,0.60,0.02}{#1}}
\newcommand{\StringTok}[1]{\textcolor[rgb]{0.31,0.60,0.02}{#1}}
\newcommand{\VariableTok}[1]{\textcolor[rgb]{0.00,0.00,0.00}{#1}}
\newcommand{\VerbatimStringTok}[1]{\textcolor[rgb]{0.31,0.60,0.02}{#1}}
\newcommand{\WarningTok}[1]{\textcolor[rgb]{0.56,0.35,0.01}{\textbf{\textit{#1}}}}
\usepackage{graphicx,grffile}
\makeatletter
\def\maxwidth{\ifdim\Gin@nat@width>\linewidth\linewidth\else\Gin@nat@width\fi}
\def\maxheight{\ifdim\Gin@nat@height>\textheight\textheight\else\Gin@nat@height\fi}
\makeatother
% Scale images if necessary, so that they will not overflow the page
% margins by default, and it is still possible to overwrite the defaults
% using explicit options in \includegraphics[width, height, ...]{}
\setkeys{Gin}{width=\maxwidth,height=\maxheight,keepaspectratio}
\IfFileExists{parskip.sty}{%
\usepackage{parskip}
}{% else
\setlength{\parindent}{0pt}
\setlength{\parskip}{6pt plus 2pt minus 1pt}
}
\setlength{\emergencystretch}{3em}  % prevent overfull lines
\providecommand{\tightlist}{%
  \setlength{\itemsep}{0pt}\setlength{\parskip}{0pt}}
\setcounter{secnumdepth}{5}
% Redefines (sub)paragraphs to behave more like sections
\ifx\paragraph\undefined\else
\let\oldparagraph\paragraph
\renewcommand{\paragraph}[1]{\oldparagraph{#1}\mbox{}}
\fi
\ifx\subparagraph\undefined\else
\let\oldsubparagraph\subparagraph
\renewcommand{\subparagraph}[1]{\oldsubparagraph{#1}\mbox{}}
\fi

%%% Use protect on footnotes to avoid problems with footnotes in titles
\let\rmarkdownfootnote\footnote%
\def\footnote{\protect\rmarkdownfootnote}

%%% Change title format to be more compact
\usepackage{titling}

% Create subtitle command for use in maketitle
\providecommand{\subtitle}[1]{
  \posttitle{
    \begin{center}\large#1\end{center}
    }
}

\setlength{\droptitle}{-2em}

  \title{Approximate Bayesian Inference for Semi-parametric Proportional Hazard
Models}
    \pretitle{\vspace{\droptitle}\centering\huge}
  \posttitle{\par}
    \author{Ziang Zhang}
    \preauthor{\centering\large\emph}
  \postauthor{\par}
      \predate{\centering\large\emph}
  \postdate{\par}
    \date{11/09/2019}


\begin{document}
\maketitle

\hypertarget{survival-analysis-model}{%
\section{Survival Analysis Model:}\label{survival-analysis-model}}

\hypertarget{introduction-to-survival-analysis}{%
\subsection{Introduction to Survival
Analysis:}\label{introduction-to-survival-analysis}}

Survival analysis refers to situations in which the response variable of
interest is the time until the occurrence of a particular event.
Examples include time to death of patients with a specific kind of
disease, time to failure of a lightbulb, add one more example.. Models
for analyzing time to event data quantify the association between
treatments or risk factors and the time to an event. For example, we may
quantify any relationship between the lifetimes of patients with the
types of medicine they are using, to conclude whether a certain type of
medicine is associated with a change in the overall survival times of
patients.

Let T be a continuous non-negative random variable representing the time
to some event, defined over the interval \([0,\infty)\). Let the
probability density function of T be denoted as \(f(t)\) and its
cumulative distribution function be \(F(t)\). The survivor function
\(S(t)\) of T can be defined as:
\begin{equation}\begin{aligned}\label{eqn:survivor}
S(t) = P(T > t) = \int_{t}^{\infty} f(x) dx = 1- F(t)
\end{aligned}\end{equation} Notice that \(S(t)\) is the probability of
an observation to survive to time t, and therefore it is a monotone
decreasing function with \(S(0) = 1\) and
\(S(\infty) = \lim_{t\to\infty} S(x) = 0\).

The hazard function, denoted h(t), is defined as instantaneous rate of
occurrence at a specific time t given that the event does not occur
before t, which can be written as:
\begin{equation}\begin{aligned}\label{eqn:hazard}
h(t) = \lim_{s\to 0} \frac{P(t\le T \le t+s |T\ge t)}{s} = \frac{f(t)}{S(t)} = -\frac{\partial}{\partial t}\text{log}[S(t)]
\end{aligned}\end{equation}

The corresponding cumulative hazard function \(H(t)\) will be defined
by: \begin{equation}\begin{aligned}\label{eqn:chazard}
H(t) = \int_{0}^{t} h(u) du = -\text{log}[S(t)]
\end{aligned}\end{equation} Cumulative hazard function is an alternative
measure of risk of occurrence, and will be more convenient to use in
some cases.

Event times are often only partially observed. Right censoring occurs
when it is only known that an event occurred after some fixed timepoint,
such as patients in a medical study who are lost to follow up. Left
truncation occurs when event times are only included in the dataset
conditional on having surpassed some threshold, such as patients joining
a medical study only after having a disease for several weeks or months.
Interval censoring refers to the presence of both right censoring and
left truncation in the same observed event times. Partially observed
observations present challenges in the analysis of survival data, and
are too common to be ignored. We expand on these challenges in Section
1.2.

\hypertarget{types-of-censoring-and-truncation}{%
\subsection{Types of Censoring and
Truncation:}\label{types-of-censoring-and-truncation}}

In survival analysis, we are mainly dealing with the problem of
right-censoring, interval-censoring and left truncation. Right-censoring
is when an individual's lifetime \(T_i\) is not exactly known because
the individual is still alive when the study terminates at \(C_i\), so
we are only sure about that \(T_i > C_i\) but not sure what exactly
\(T_i\) is. Interval-censoring on the other hand, rises when the
survival time is only known to be in an interval \((L_i,U_i)\), and the
left truncation problem happens when some survival times are not
recorded unless they are bigger than a specified start time \(t^{tr}\),
so all the data with survival times less than \(t^{tr}\) are missed.

In general, we use the term ``censoring'' to refer to the scenarios
where some lifetimes are only known to exceed their cutting times
\(C_i\), but we do not know how long do they last exactly. On the other
hand, the term ``truncation'' mostly refer to a data collected problem
where only lifetimes greater than the start time \(t^{tr}\) are
collected and observed. So these terms should be used in different
situations depending on what kind of survival data are we dealing with.

There are two types of right-censoring that appear most frequently in
the context of survival analysis, which are Type-I and Type-II
right-censoring.

Type-I right-censoring occurs when each individual's censoring time
\(C_i\) is fixed and known beforehand. That means when we collect a
bunch of survival times, we know whether each survival time is
right-censored and when is it censored exactly. For example, in a
medical study, we may give each individual 30 minutes of observation
time after each of them took a specific type of medicine, and the
\(C_i\) in this case is fixed to be 30 minutes for all i.

For this type of censoring mechanism, we will be able to write our
original data-set \(\{ T_i,C_i:i=1,...,n \}\) as
\(\{ t_i,\delta_i:i=1,...,n \}\) where:

\begin{equation}\begin{aligned}\label{eqn:transformed data}
t_i = \text{min}\{T_i,C_i \},  \qquad  \delta_i = I(T_i \leq C_i)
\end{aligned}\end{equation}

Many statistical procedures dealing with right censoring are assuming
this is the underlying censoring mechanism (Kwan, 1997), so we will
focus on this type of censoring for the rest of the paper.

Type-II right-censoring occurs when we only observed the r smallest
survival times in our sample. So the survival times that we can observed
will be like \(t_{(1)}<t_{(2)}<...<t_{(r)}\), and the other survival
times will be censored so we don't know the exact numbers. In this
scenario, we have a censoring time \(t_{(r)}\) that is itself random.
For example, in an engineering study, we can measure the length of time
for each component to become defective, and we will end the study after
collecting ten defectives in our sample. In this case, the censoring
time will be \(t_{10}\), and all the times after this will not be
observable.

Lastly, independent random censoring happens when both the ith survival
time \(T_i\) and the ith censoring time \(C_i\) are random variable that
are independent.For example, if we want to measure the lifetimes of
patients in a hospital, then it is possible that some patients are going
to switch to anther hospital before the study ends. In this case, if we
assume that switching hospital does not have relationship with the
severity of the deterioration of diseases, then the censoring times will
be independent with the lifetimes of patients.

\newpage

\hypertarget{cox-proportional-hazard-model}{%
\subsection{Cox Proportional Hazard
Model:}\label{cox-proportional-hazard-model}}

In most survival analysis study, we are interested in incorporating some
covariates \(\tilde{X} =\{X_1,X_2,...,X_p\}\) into the distribution of
survival time \(T\), and studying their associations with the survival
time \(T\).Define \(u_{qi}\) be the covariates that are modelled
semi-parametrically using unknown smooth functions \(\gamma_q\).
Therefore, to specify the dependence of T on \(\tilde{X}\),the
proportional hazard model introduced by Cox(1972) is the a popular
choice.

Let \(h(t|\tilde{x})\) denote the hazard function of \(T\) at time t for
a subject with covariates \(\tilde{x} = (x_1,x_2,...,x_p)\). The Cox
Proportional Hazard Model can be specified as follows:
\begin{equation}\begin{aligned}\label{eqn:CoxHazardModel}
h(t|\tilde{x}) = h_0(t)\text{exp}(\beta_1x_1+...+\beta_px_p+\sum_{q=1}^{R} \gamma_q(u_{qi}))
\end{aligned}\end{equation} where \(h_0(t)\) is an arbitrary baseline
hazard function that does only depend on time, and \(\beta_i\)'s are the
unknown parameters that we are interested in estimating. The reason that
it is called a ``proportional'' hazard model is because for any two
subjects, the ratio of their hazard function will be constant over time.
This is a model assumption, and it should be checked before adopting the
Cox proportional hazard model.

Notice that the baseline hazard function is left to be arbitrary, which
implies that the Cox Proportional Hazard Model will be a semi-parametric
model. To specify the baseline hazard functions, we are going to use the
piece-wise constant baseline hazard model, and I will introduce it in
details in the next section.

\hypertarget{proportional-hazard-model-with-piece-wise-constant-baseline-hazard}{%
\subsection{Proportional Hazard Model with Piece-wise Constant Baseline
Hazard:}\label{proportional-hazard-model-with-piece-wise-constant-baseline-hazard}}

Firstly, we break the time axis into K intervals with endpoints
\(0=s_0<s_1<...<s_K < \text{max}\{t_i:i=1,...,n\}\), and assumes that
the baseline hazard function is constant in each interval, i.e:
\(h_0(t) = b_k\) for \(t\in(s_{k-1},s_k), \ k=1,2, ...,K\).

Let
\(\eta_{ik} = \text{log}(b_k)+\tilde{x_i}^T\tilde{\beta}+\sum_{q=1}^{R} \gamma_q(u_{qi})\),
the model that we will be focusing on will be the semi-parametric
proportional hazard model, specified at below:
\begin{equation}\begin{aligned}\label{eqn:phmodel}
h(t_i) &= h_0(t_i)\text{exp}[\tilde{x_i}^T\tilde{\beta}+\sum_{q=1}^{R} \gamma_q(u_{qi})]\\
       &= \text{exp}[\text{log}(b_k)+\tilde{x_i}^T\tilde{\beta}+\sum_{q=1}^{R} \gamma_q(u_{qi})] \qquad t_i\in(s_{k-1},s_k] \\
       &= \text{exp}(\eta_{ik})
\end{aligned}\end{equation}

Using this information, we can derive the likelihood for that single
observation to be: \begin{equation}\begin{aligned}\label{eqn:singlelike}
\pi\big[(t_i,\delta_i )  | \eta_{i1} ... \eta_{ik} \big] &= f(t_i)^{\delta_i}S(t_i)^{(1- \delta_i)}\\
  &= h(t_i)^{\delta_i}S(t_i) \\
  &= \text{exp}(\delta_i \eta_{ik})\bigg\{ \text{exp} \big[ -\int_{0}^{t_i} h(u) du\big ] \bigg\} \\
  &= \text{exp}(\delta_i \eta_{ik})\bigg \{\text{exp}\big[-\sum_{j=1}^{k-1} (s_{j}-s_{j-1})\text {exp} (\eta_{ij}) - (t_i-s_{k-1})\text{exp}(\eta_{ik})\big]\bigg \}\\
\end{aligned}\end{equation}

Therefore, the full-likelihood of the data-set will be:
\begin{equation}\begin{aligned}\label{eqn:fulllike}
L &=  \prod_{i=1}^{n} \text{exp}(\delta_i \eta_{ik_{(i)}}) \text{exp} \bigg\{-\sum_{j=1}^{k_{(i)}-1} (s_{j}-s_{j-1})\text{exp}(\eta_{ij})-(t_i - s_{k_{(i)}-1})\text{exp}(\eta_{ik_{(i)}})\bigg \} \\
  &= \prod_{i=1}^{n} \text{exp} \big \{\delta_i \eta_{ik_{(i)}} -\sum_{j=1}^{k_{(i)}-1} (s_{j}-s_{j-1})\text{exp}(\eta_{ij})-(t_i - s_{k_{(i)}-1})\text{exp}(\eta_{ik_{(i)}}) \big \}
\end{aligned}\end{equation}

I emphasize the subscript for \(k_{(i)}\) because each survival time
will correspond to a different value of \(k_{(i)}\), depending on which
interval the survival time lies in. More specifically, \(k_{(i)}\)
denotes which one of the k sub-intervals the observation is in.

By taking the logarithm, the log-likelihood function for the \(i^{th}\)
observation \(t_i \in (s_{k-1},s_k]\) can be written as :
\begin{equation}\begin{aligned}\label{eqn:loglike}
l &= \text{log}[f(t_i)^{\delta_i}S(t_i)^{(1-\delta_i)}] \\
  &= \text{log}[h(t_i)^{\delta_i}S(t_i)]\\
  &= \delta_i \eta_{ik} - (t_i-s_{k-1})\text{exp}(\eta_{ik})-\sum_{j=1}^{k-1} [(s_{j}-s_{j-1})\text{exp}(\eta_{ij})]
\end{aligned}\end{equation}

Similarly, the full log-likelihood can be derived as:
\begin{equation}\begin{aligned}\label{eqn:fullloglike1}
l = \sum_{i=1}^{n} \bigg \{ \delta_i \eta_{ik_{(i)}} - (t_i-s_{k_{(i)}-1})\text{exp}(\eta_{ik_{(i)}})-\sum_{j=1}^{{k_{(i)}}-1} [(s_{j}-s_{j-1})\text{exp}(\eta_{ij})]\bigg \}
\end{aligned}\end{equation}

It can see from the above expression that by considering a piece-wise
constant baseline hazard, we make the corresponding log-likelihood much
easier to work with, since the integral \(\int_{0}^{t_i} h(u)du\) can be
replaced by a sum.

\hypertarget{inlas-inference-methodology}{%
\section{INLA's Inference
Methodology:}\label{inlas-inference-methodology}}

In this section, I will briefly introduce what INLA actually does to do
the approximate Bayesian inference for Cox proportional hazard model,
and show what are the current limitations of INLA for this type of
problem.

\hypertarget{data-augmentation-using-poisson-likelihood}{%
\subsection{Data Augmentation Using Poisson
Likelihood:}\label{data-augmentation-using-poisson-likelihood}}

For Cox proportional hazard model, the INLA algorithm cannot directly be
applied, because if we look at the log-likelihood of a single survival
time \(\{t_i,\delta_i\}\), we can find that it depends on more than one
\(\eta\). To use INLA, we required a conditional independent latent
field together with a sparse Hessian matrix for the log-likelihood. That
means we need to make sure that for a single data point, the
log-likelihood should be free of terms from latent field once we
condition on one of the term from the latent field.

To solve this puzzle, INLA utilizes a data ``augmentation'' trick to
transform the log-likelihood of a single data point into the form that
INLA likes. Notice that if we are looking at a random variable \(X_i\)
that follows a Poisson distribution with mean
\((t_i-s_{k-1})\text{exp}(\eta_{ik})\), then the log-likelihood
corresponding to a single data point \(\{X_i = 0 \}\) will be:
\begin{equation}\begin{aligned}\label{eqn:loglike1}
l &= \text{log}\bigg \{P\big [X_i =0|\lambda = (t_i-s_{k-1})\text{exp}(\eta_{ik}) \big]\bigg \}\\
  &= 0\times \text{ln}[(t_i-s_{k-1})\text{exp}(\eta_{ik})] - (t_i-s_{k-1})\text{exp}(\eta_{ik}) - \text{ln}(0!)\\
  &= - (t_i-s_{k-1})\text{exp}(\eta_{ik})
\end{aligned}\end{equation} Similarly, when \(X_i = 1\), the
log-likelihood of this single data point is:
\begin{equation}\begin{aligned}\label{eqn:loglike2}
l &= \text{log}\bigg(P(X_i =1|\lambda = (t_i-s_{k-1})\text{exp}(\eta_{ik}))\bigg)\\
  &= 1\times \text{ln}((t_i-s_{k-1})\text{exp}(\eta_{ik})) - (t_i-s_{k-1})\text{exp}(\eta_{ik}) - \text{ln}(1!)\\
  &= \text{ln}(t_i-s_{k-1})+\eta_{ik}-(t_i-s_{k-1})\text{exp}(\eta_{ik})\\
  &\propto \eta_{ik}-(t_i-s_{k-1})\text{exp}(\eta_{ik})
\end{aligned}\end{equation}

Here we can basically ignore the term \(\text{ln}(t_i-s_{k-1})\) as it
does not depend on any term from the latent field. So when we later take
derivative, this term will just disappear which means it won't affect
our C matrix.

We showed that the first two terms of the log-likelihood of a single
data point \(\{t_i,\delta_i\}\) can be viewed as the log-likelihood of a
single data point
\(X_i\sim \text{Poisson}\big[\lambda =(t_i-s_{k-1})\text{exp}(\eta_{ik})\big]\)
being \(0\) when \(\delta_i = 0\) and being 1 when \(\delta_i = 1\).

Next step will be to figure out a similar way to deal with the last term
in equation (3). Notice that for a Poisson random variable \(Y_j\) with
mean \((s_{j}-s_{j-1})\text{exp}(\eta_{ij})\), the log-likelihood for
observing it being \(0\) will be:
\begin{equation}\begin{aligned}\label{eqn:loglike3}
l &= \text{log}\bigg \{P\big [Y_j =0|\lambda = (s_{j}-s_{j-1})\text{exp}(\eta_{ij})\big ]\bigg \}\\
  &= -(s_{j}-s_{j-1})\text{exp}(\eta_{ij})
\end{aligned}\end{equation}

Similarly, if we gather a sample of
\(\{Y_{i_{1}}=0,Y_{i_{2}} =0, ..., Y_{i_{k}}=0 \}\) where each
\(Y_{i_j} \sim \text{Poisson}\big[\lambda = (s_{j}-s_{j-1})\text{exp}(\eta_{ij})\big]\)
is independent of others, then the log-likelihood of this sample will
simply be the sum of log-likelihood of each term due to independence,
which sums to be
\(\sum_{j=1}^{k-1} (s_{j}-s_{j-1})\text{exp}(\eta_{ij})\),that is
exactly what we want.

Putting these two pieces information together, which means if we have a
sample being
\(\{X_i =\delta_i,Y_{i_{1}}=0,Y_{i_{2}} =0, ..., Y_{i_{k}}=0 \}\), and
all the terms in this sample being mutually independent, then the
log-likelihood of this sample will just be the log-likelihood of the
single data point \(\{t_i,\delta_i\}\). Doing this for all the data
points \(\{t_i,\delta_i|i=1,...,n\}\). We retrieve the original
log-likelihood from the log-likelihood of a sample of
\(\sum_{i=1}^{n}{k_{(i)}}\) number of independent, but non-identical
Poisson random variables. In other words, we augment our original
data-set \(\{t_i,\delta_i|i=1,...,n\}\) into a huge
data-set\(\{x_i ,y_{i_{1}},y_{i_{2}}, ..., y_{i_{k_{(i)}}}|i=1,2,...n\}\),
where all the terms in this new data-set are mutually independent. This
is the cure for our problem since the log-likelihood of each term from
this new ``augmented'' data-set, will only depend on the latent field
through one \(\eta\).

\hypertarget{derivation-of-the-negated-hessian-matrix}{%
\subsection{Derivation of the Negated Hessian
Matrix:}\label{derivation-of-the-negated-hessian-matrix}}

Here I will present how the Bayesian approximation can be carried out
using an INLA-type of algorithm. Firstly, to make the covariance matrix
of the joint Gaussian latent field non-singular, and to simplify the
Hessian matrix that we are going to derive later, we will assume that
for each \(\eta_{ij}\), a normal random noise \(\epsilon_{ij}\) is
added. We assume that \(\epsilon_{ij} \sim N(0,\sigma^2_{\epsilon})\)
being mutually independent across different i and j. In other words, we
will write
\(\eta_{ik} = \text{log}(b_k)+\tilde{x_i}^T\tilde{\beta}+\sum_{q=1}^{R} \gamma_q(u_{qi})+\epsilon_{ik}\).
Let \(U_q = \{U_{ql},l=1,...,M_q\}\) be an ordered vector of all unique
values of \(u_{ql}\). Define
\(\Gamma_q = \{\gamma_{q}(U_{ql}), l=1,...,M_q\}\), and
\(\Gamma = \{\Gamma_q, q=1,...,R\}\).

Then, the latent field can be denoted as:
\begin{equation}\begin{aligned}\label{eqn:field1}
\tilde W = \big[\eta_{11},\eta_{12},...,\eta_{1k},\eta_{2k},...,\eta_{nk},\Gamma_1,...,\Gamma_R,\beta_1,...,\beta_p,\text{log}(b_1),...,\text{log}(b_k)\big]^T\\
\end{aligned}\end{equation} which has dimension
\(nk+\sum_{q=1}^{R}M_q + p +k\).

Besides assume that \(\tilde W\) is a Gaussian Markov Random Field, we
also assume that \(\text{log}(b_{k+1})-\text{log}(b_k)\) follows
\(N(0,\tau^{-1})\), a random walk with order 1. So we will just use
\(\tilde \theta\) to denote the hyper-parameter vector that determines
the precision matrix of our latent field.

Now, let's derive the negated Hessian matrix of the log-likelihood with
respect to the latent field. To do that, let's first consider the
log-likelihood consider only one survival time \(\{t_i,\delta_i\}\)
where \(t_i \in (s_{k_{(i)}-1},s_{k_{(i)}}]\). In this case, the
log-likelihood for this data point will be:
\begin{equation}\begin{aligned}\label{eqn:loglikeagain}
l = \delta_i \eta_{ik_{(i)}} - (t_i-s_{k_{(i)}-1})\text{exp}(\eta_{ik_{(i)}})-\sum_{j=1}^{k_{(i)}-1} [(s_{j}-s_{j-1})\text{exp}(\eta_{ij})]
\end{aligned}\end{equation} The derivative with respect to
\(\eta_{ik_{(i)}}\) will be\\
\begin{equation}\begin{aligned}\label{eqn:hessian}
\frac{\partial l}{\partial \eta_{ik_{(i)}}}= \delta_i -(t_i-s_{k_{(i)}-1})\text{exp}(\eta_{ik_{(i)}})
\end{aligned}\end{equation} That means the negated second derivative
will be: \begin{equation}\begin{aligned}\label{eqn:hessian1}
-\frac{\partial^2 l}{\partial {\eta_{ik_{(i)}}}^2} = (t_i-s_{k_{(i)}-1})\text{exp}(\eta_{ik_{(i)}})
\end{aligned}\end{equation}

For first and negated second derivatives with \(\eta_{ij}\) where
\(j<k_{(i)}\), we have:
\begin{equation}\begin{aligned}\label{eqn:hessian2}
\frac{\partial l}{\partial \eta_{ij}}= -(s_{j}-s_{j-1})\text{exp}(\eta_{ij})\\
-\frac{\partial^2 l}{\partial {\eta_{ij}}^2} = (s_{j}-s_{j-1})\text{exp}(\eta_{ij})
\end{aligned}\end{equation}

Apparently, for \(\eta_{ij}\) where \(j>k_{(i)}\), we have the second
derivatives of log-likelihood being 0's. Combine them together, we know
that the negated Hessian matrix for the log-likelihood of
\(\{t_i,\delta_i\}\), \(C_i\) will be: \begin{equation}
\begin{bmatrix}
(s_1-s_0)\text{exp}(\eta_{i1})  & 0  & 0 & \cdots & \cdots & \cdots & 0 \\
0  & (s_2-s_1)\text{exp}(\eta_{i2})  & 0  & \ddots & && &  \\
0 & 0  & \ddots &   & \ddots & &  &  \\
\vdots & \cdots & \cdots & (s_{k_{(i)}-1}-s_{k_{(i)}-2})\text{exp}(\eta_{i(k_{(i)}-1)}) & \ddots & \vdots &  &  \\
\vdots & & \ddots & 0 & (t_i-s_{k_{(i)}-1})\text{exp}(\eta_{ik_{(i)}}) & \cdots & \vdots& \\
\vdots  & & & \ddots &   & \ddots  &  \vdots\\
\vdots  & && & \cdots & \cdots & \vdots\\
0 & \cdots &  \cdots & \cdots & \cdots & \cdots & 0\\
\end{bmatrix}
\end{equation} This is a very sparse matrix with only diagonal terms. It
is a \(k\times k\) matrix, but only the diagonal terms at the first
\(k_{(i)}\) rows are non-zero.

Repeating this procedure for the rest data points, using the property of
independence, we can get the negated Hessian matrix C for the full
log-likelihood will be: \begin{equation}
C = \begin{pmatrix} 
C_1 & 0 & 0 & \cdots & & \\ 
0 & C_2 & 0 & \cdots & & \\
  & \cdots & \ddots &  & & \\
& & & & C_n & \cdots & \vdots \\ 
& & & \ddots & &&\vdots \\
& & & & & & 0
\end{pmatrix}
\end{equation}

Here we build a block diagonal matrix C using each block \(C_i\)
obtained from above procedures. The negated Hessian matrix is sparse and
diagonal, which is exactly what it has to be to be fitted using INLA.
This full matrix has dimension \((nk+\sum_{q=1}^{R}M_q + p +k)\)
\(\times\) \((nk+\sum_{q=1}^{R}M_q + p +k)\), and each block matrix
\(H_{i}\) has different patterns of zeros but same dimension.

However, in a lot of cases, the data augmentation trick that INLA does
will not work, and the negated Hessian matrix is not diagonal. For
example, if we want to only use the partial likelihood instead of the
full likelihood, because the baseline function is not of interest, then
the resulting negated Hessian matrix will be sparse but not diagonal.
INLA is currently not applicable to this type of problem, and here we
are suggesting a novel method that will be general enough to be
applicable even when the data augmentation trick cannot work.

\hypertarget{proposed-methodology-for-approximation}{%
\section{Proposed Methodology for
Approximation:}\label{proposed-methodology-for-approximation}}

In the paper ``Approximate Bayesian Inference for Case-Crossover
Models'', Stringer et al.(2019) suggested a new type of algorithm to do
the approximation while allowing the log-likelihood of each observation
to be dependent on more than one element from the latent field, which
means the ad-hoc method using ``data augmentation'' is no longer needed.
Here we will demonstrate how that algorithm can be used to estimate the
parameters in Cox Proportional Hazard Model, and when this new algorithm
will be preferred than INLA's algorithm.

\hypertarget{approximation-using-partial-likelihood-with-right-censoring-only}{%
\subsection{Approximation using Partial Likelihood with Right censoring
only:}\label{approximation-using-partial-likelihood-with-right-censoring-only}}

For simplicity, let's assume that our main interests are the components
in the linear predictor \(\eta\), but not the baseline hazard
\(h_0(t)\), and the only type of censoring present is right-censoring.
Let the C-matrix denote the negated Hessian matrix of the log-likelihood
with respects to the latent field \(\tilde{W}\).

Assume that \(\{t_i:i=1,...,k\}\) is a set of k distinct lifetimes that
we actually \(observed\), such that
\(t_{(1)} < t_{(2)} < ... <t_{(k)}\), and the result n-k lifetimes are
the censored lifetimes that are not observed. Let \(R_i = R(t_{(i)})\)
be the set of individuals who are alive and uncensored prior to time
\(t_{(i)}\) (including the i-th individual who dies at \(t_{(i)}\)).

Define the hazard function for the i-th individual to be
\(h_0(t)\text{exp}(\eta_i)\), and let \(\Delta_{i,j}\) =
\(\eta_i -\eta_j\), then the partial likelihood for Cox Proportional
Hazard Model can be written as:
\begin{equation}\begin{aligned}\label{eqn:partial}
L^{partial} &= \prod_{i=1}^{k} \bigg\{\frac{\text{exp}[\eta_{(i)}]}{{\sum_{l\in R_i}^{}\text{exp}[\eta_{(l)}]}}\bigg \} \\
         &= \prod_{i=1}^{k} \bigg\{\frac{1}{{\sum_{l\in R_i}^{}\text{exp}[\eta_{(l)}-\eta_{(i)}]}}\bigg \} \\
         &= \prod_{i=1}^{k} \bigg\{\frac{1}{{\sum_{l\in R_i}^{}\text{exp}[\Delta_{l,i}]}}\bigg \} \\
         &= \prod_{i=1}^{k} \bigg\{\frac{1}{{1 + \sum_{l\in R_i , l \neq i}^{}\text{exp}[\Delta_{l,i}]}}\bigg \} \\
\end{aligned}\end{equation}

Which can be written as:
\begin{equation}\begin{aligned}\label{eqn:partialagain}
&= \prod_{i=1}^{k} \bigg\{\frac{1}{{1 + \sum_{l\in R_i , l \neq i}^{}\text{exp}[(\Delta_{1,i}-\Delta_{1,l})]}}\bigg \} \\
\end{aligned}\end{equation}

Because: \begin{equation}\begin{aligned}\label{eqn:partialagain2}
\Delta_{i,j} = \eta_i-\eta_j = \eta_1-\eta_j-(\eta_1-\eta_i)=\Delta_{1,j}-\Delta_{1,i} 
\end{aligned}\end{equation} \newline

Define \(\Delta:=(\Delta_{1,2},\Delta_{1,3},...,\Delta_{1,n})^T\), which
are all the \(\Delta_{ij}\)'s that will contribute to the likelihood.
\newline

Notice that this partial likelihood does not include any information on
the baseline hazard function \(h_0(t)\), meaning that all of the
information are used to estimate the regression parameters in the model,
which should result in a more precise estimation for them. Here it is
shown that the partial likelihood only depend on those ``differenced
linear predictors'' \(\{\Delta_{1,i}-\Delta_{1,j}\}\), so our latent
field in this case will be \(\{ \Delta,\beta,\Gamma \}\). More
importantly, because we are not estimating those baseline hazards, the
algorithm's convergence rate is expected to be faster. INLA does not
allow this type of approximation because using partial likelihood to
ignore the baseline hazard invalidates the ``Poisson data-augmentation''
trick that INLA does to make the C-matrix diagonal. While non-diagonal C
matrix is not feasible in INLA's algorithm, it will be feasible in the
new proposed algorithm.

\hypertarget{derivation-of-hessian-matrix-and-precision-matrix}{%
\subsubsection{Derivation of Hessian matrix and Precision
matrix:}\label{derivation-of-hessian-matrix-and-precision-matrix}}

For i-th observation, the partial log-likelihood will be:
\begin{equation}\begin{aligned}\label{eqn:ithPart}
l &= -\text{ln}(1+{\sum_{j\in R_i, j \ne i}^{}\text{exp}[\Delta_{j,i}]}) \\
  &= -\text{ln}(1+{\sum_{j\in R_i, j \ne i}^{}\text{exp}[(\Delta_{1,i}-\Delta_{1,j})]})\\
\end{aligned}\end{equation}

Therefore, taking derivative with respect to \(\Delta_{1,i}\), we can
get: \begin{equation}\begin{aligned}\label{eqn:derivofPart}
\frac{\partial l}{\partial \Delta_{1,i}} &= -\frac{\sum_{j\in R_i, j \ne i}^{}{}\text{exp}[(\Delta_{1,i}-\Delta_{1,j})]}{1+\sum_{j\in R_i, j \ne i}^{}{}\text{exp}[(\Delta_{1,i}-\Delta_{1,j})]} \\
\end{aligned}\end{equation}

Similarly, we can see that:

\begin{equation}\begin{aligned}\label{eqn:secderivofPart}
\frac{\partial^2 l}{\partial \Delta_{1,i}^2} &= -\frac{\sum_{j\in R_i, j \ne i}^{}{}\text{exp}[(\Delta_{1,i}-\Delta_{1,j})]}{\big \{1+\sum_{j\in R_i, j \ne i}^{}{}\text{exp}[(\Delta_{1,i}-\Delta_{1,j})]\big \}^2 }
\\
\end{aligned}\end{equation}

Suppose that \(w\ne i\), if \(w\notin R_i\) then we have:
\begin{equation}\begin{aligned}\label{eqn:easyderiveforwnotini}
\frac{\partial l}{\partial \Delta_{1,w}} = 0
\end{aligned}\end{equation}

If \(w\in R_i\) then we have:
\begin{equation}\begin{aligned}\label{eqn:secderivofPart3}
\frac{\partial l}{\partial \Delta_{1,w}} = \frac{\text{exp}[(\Delta_{1,i}-\Delta_{1,w})]}{1+\sum_{j\in R_i, j \ne i}^{}{}\text{exp}[(\Delta_{1,i}-\Delta_{1,j})]}
\end{aligned}\end{equation}

Similarly, we get the following result for \(w\in R_i\):
\begin{equation}\begin{aligned}\label{eqn:result_forsecdr}
\frac{\partial^2 l}{\partial \Delta_{1,i} \partial \Delta_{1,w}} = \frac{\text{exp}(\Delta_{1,i}-\Delta_{1,w})}{\big\{1+\sum_{j\in R_i, j \ne i}^{}{}\text{exp}[(\Delta_{1,i}-\Delta_{1,j})]\big \}^2}
\end{aligned}\end{equation}

\begin{equation}\begin{aligned}\label{eqn:result_forsecdr2}
\frac{\partial^2 l}{\partial \Delta_{1,w}^2} = -\frac{\big[\sum_{j \in R_i, j \neq i,w}^{} \text{exp} (2\Delta_{1,i}-\Delta_{1,w}-\Delta_{1,j}) \big] + \text{exp}(\Delta_{1,i}-\Delta_{1,w})}{\big[1+\text{exp}(\Delta_{1,i}-\Delta_{1,w})\big]^2}
\end{aligned}\end{equation}

If \(M \in R_i\) but \(M\neq i, M\neq w\), then:
\begin{equation}\begin{aligned}\label{eqn:result_forsecdrlast}
\frac{\partial^2 l}{\partial \Delta_{1,w} \partial \Delta_{1,M}} = \frac{\text{exp}(2\Delta_{1,i}-\Delta_{1,w}-\Delta_{1,M})}{\big\{1+\sum_{j\in R_i, j \ne i}^{}{}\text{exp}[(\Delta_{1,i}-\Delta_{1,j})]\big \}^2}
\end{aligned}\end{equation}

From the information above, we can derive the negated Hessian matrix of
log-likelihood of the i-th observation, with respect to \(\Delta\)
being:

\begin{equation}
C_{i} = -\begin{pmatrix}
\frac{\partial^{2}l}{\partial \Delta_{1,2}^{2}} & \frac{\partial^{2}l}{\partial \Delta_{1,2}\Delta_{1,3}} & \cdots & \frac{\partial^{2}l}{\partial \Delta_{1,2}\Delta_{1,n}} \\
\frac{\partial^{2}l}{\partial \Delta_{1,3}\Delta_{1,2}} & \frac{\partial^{2}l}{\partial \Delta_{1,3}^2} & \cdots & \vdots \\
\vdots & \vdots & \ddots & \vdots \\
\frac{\partial^{2}l}{\partial \Delta_{1,n}\Delta_{1,2}} & \cdots & \cdots & \frac{\partial^{2}l}{\partial \Delta_{1,n}^2}
\end{pmatrix}
\end{equation}

For each \(i \in\{1,...,n\}\), the corresponding \(C_i\) matrix will be
\((n-1)\times(n-1)\). Since the log-likelihood of the sample will be the
sum of log-likelihood of each data, the negated Hessian matrix
\(C^\Delta\) of the full log-likelihood with respect to \(\Delta\) will
be the sum of each \(C_i\) matrix, i.e.: \(C^\Delta = C_1 + ...+C_k\).

Therefore, the negated Hessian matrix C of the full log-likelihood with
respect to the whole latent field will be:

\begin{equation}\label{eqn:thehessian}
C = \begin{pmatrix}
C^{\Delta} & & & & & \\
& 0 & & & & \\
& & 0 & & & \\
& & & 0 & & \\
& & & & \ddots & & \\
& & & & & 0 \\
\end{pmatrix}
\end{equation}

To get the vector
\(\Delta=(\Delta_{1,2},\Delta_{1,3},..., \Delta_{1,n})^T\), we need to
apply a linear transformation on the vector
\(\eta=(\eta_1,...,\eta_n)^T\), in other words:

\begin{equation}\begin{aligned}\label{eqn:D1}
\Delta &= \begin{pmatrix}\eta_{1} - \eta_{2} \\ \vdots \\ \eta_{1} - \eta_{n} \end{pmatrix} = D\eta \\
\end{aligned}\end{equation} Where \(D\) is the differencing matrix
defined as: \begin{equation}\begin{aligned}\label{eqn:D2}
D &= \begin{pmatrix}
1 & -1 & 0 & \cdots & 0 \\
1 & 0 & -1 & \cdots & 0 \\
  &    & \ddots &   &   \\
1 &    &       & 0 & -1 \\
\end{pmatrix}
\end{aligned}\end{equation} In this case, D will be a matrix with
dimension \((n-1)\times n\) and rank \(n-1\). Therefore, \(DD^T\) will
be an invertible matrix.

Using the differenced matrix D defined above, we can proceed to get the
precision matrix Q. The rest procedures will be identical to the
procedures Stringer did in the paper ``Approximate Bayesian Inference
for Case-Crossover Models'' (Stringer et al,2019).

\hypertarget{adjustments-of-likelihood-for-tied-observations}{%
\subsubsection{Adjustments of Likelihood for tied
observations}\label{adjustments-of-likelihood-for-tied-observations}}

Throughout our discussion on how to use partial likelihood for doing
approximate Bayesian Inference, we made an important assumption that the
uncensored lifetimes are all unique, in other words
\(t_{(i)} \ne t_{(j)}\) for all \(i\neq j\). However, in real life
applications, there will be some cases where this assumption is not met.
For example, we might use a ``discretized'' measure of lifetime when we
are recording both 12.11 hours and 12.12 hours as 12.1 hours in the
study of sustainability of a certain type of battery. That means, we
should be able to deal with ``ties'' in our observed lifetimes. Here we
will focus on Efron and Breslow's methods of approximation.

From now, let's assume that for an observed lifetime \(t_j\), there are
\(d_{(j)}\) number of other observed lifetimes are ``tied'' with it. Let
\(D_{(j)}\) be the set of all the individuals with lifetimes tied at
\(t_j\), that means there will be \(d_{(j)}\) number of elements in
\(D_{(j)}\).

Breslow's method of approximation will be the easiest and fastest method
to use, when accuracy of estimations is not extremely required. The main
idea of Breslow approximation is to include all the events occured at
\(t_j\) in the risk set \(R(t_j)\). Therefore, we can write the partial
likelihood at \(t_j\) as:

\begin{equation}\begin{aligned}\label{eqn:tie1forBres}
L_j^B = \frac{\prod_{i \in D_{(j)}} \text{exp}(\eta_i)}{\big[ \sum_{i\in R_{j}} \text{exp}(\eta_i) \big]^{d_{(j)}}}
\end{aligned}\end{equation}

The use of Breslow method is really convenient, because its partial
likelihood has the same form of our ordinary partial likelihood when all
the observations are distinct. When the number of ties in the data-set
is not a lot, Breslow's method does give a fairly accurate result, and
its computation time is exactly the same as no ties in the set. However,
despite its appealing efficiency, if too many ties occuring at the same
time, then its accuracy will be influenced comparatively heavily. In
that case, we would like to consider an alternative method, which is
Efron method of approximation.

Efron initially suggested this method as an approximation for the case
when these ties are ``exactly continuous''. However, Efron's method in
general gives very accurate estimates, and is computationally fast
enough, even when the number of ties is very large. The main idea of
this approach is to give a partial approximation to the contribution to
\(R(t_j)\) of each tied occurence, and it uses a partial likelihood in
the form below:

\begin{equation}\begin{aligned}\label{eqn:tie1forEf}
L_j^{Ef} = \frac{\prod_{i \in D_{(j)}} \text{exp}(\eta_i)}{\prod_{h = 1}^{d_{(j)}} \bigg \{\sum_{i\in R_{j}} \text{exp}(\eta_i) - \frac{h-1}{d_j}\sum_{k\in D_{j}} \text{exp}(\eta_k) \bigg \}}
\end{aligned}\end{equation}

Although the partial likelihood of Efron's approximation is not as easy
as the partial likelihood of Breslow's method, it is still not bad for
our computational efficiency. In the case of a lot of ties occuring at a
specific time, it gives a more accurate result than Breslow's method.

There are other exact methods to deal with ties such as D.Cox's ``exact
discrete'' method and Kalbfleisch and Prentice's ``exact continuous''
method, but they are computationally too heavy to be used for our
purpose. So we choose to not use them for now.

\hypertarget{adjustment-to-c-matrix-using-breslows-method}{%
\paragraph{Adjustment to C-matrix using Breslow's
Method:}\label{adjustment-to-c-matrix-using-breslows-method}}

If Breslow's method is used to adjust for the effect of tied
observations at time \(t_{(j)}\), then the form of C-matrix will be very
similar to the C-matrix under regular setting derived in section 3.1.1.
However, there will be some small changes we need to add to the
C-matrix.

Firstly, the partial likelihood at time \(t_{(j)}\) will be:
\begin{equation}\begin{aligned}\label{eqn:moreonBresl}
L_j^B = \prod_{i\in D_{(j)}}^{}\bigg[ \frac{1}{\sum_{k\in R_j}^{}\text{exp}(\Delta_{k,i})} \bigg]
\end{aligned}\end{equation}

Therefore, its logarithm will be:
\begin{equation}\begin{aligned}\label{eqn:moreonBresl2}
l_j^B &= \sum_{i\in D_{(j)}}^{}\text{log}\bigg[ \frac{1}{1+\sum_{k\in R_j,k\neq i}^{}\text{exp}(\Delta_{k,i})} \bigg]\\
&= -\sum_{i\in D_{(j)}}\text{log}\bigg[ {1+\sum_{k\in R_j,k\neq i}^{}\text{exp}(\Delta_{k,i})} \bigg]\\
&= -\sum_{i\in D_{(j)}}\text{log}\bigg[ {1+\sum_{k\in R_j,k\neq i}^{}\text{exp}(\Delta_{1,i}-\Delta_{1,k})} \bigg]\\
\end{aligned}\end{equation}

That means, for the subject \(i\) such that \(i\in D_{(j)}\), we have
its log-likelihood being:
\begin{equation}\begin{aligned}\label{eqn:moreonBresl2fori}
l_{j(i)}^B = -\text{log}(1+\sum_{k\in R_{j},k\neq i}\text{exp}(\Delta_{1,i}-\Delta_{1,k}))
\end{aligned}\end{equation} The negated Hessian matrix \(C_{j_{(i)}}\)
of this log-likelihood with respect to \(\Delta\) can be computed using
the same procedures in section 3.1.1.

Consequently, the negated Hessian matrix of the log-likelihood at time
\(t_{(j)}\) will be:
\begin{equation}\begin{aligned}\label{eqn:moreonBresllast}
C_j = \sum_{i \in D_{(j)}}C_{j_{(i)}}
\end{aligned}\end{equation}

Similarly, we are able to get negated Hessian matrix \(C_i\) at each
observed time point \(t_{(i)}\), and combing them using the method from
section 3.1.1 gives our final C matrix of the sample.

\hypertarget{adjustment-to-c-matrix-using-efrons-method}{%
\paragraph{Adjustment to C matrix using Efron's
Method:}\label{adjustment-to-c-matrix-using-efrons-method}}

Define the vector \(\tilde{h}=(1,2,...,d_{(j)})^T\), where \(h_i\) is
its \(i\)th component, then the partial likelihood at time \(t_{(j)}\)
can be written as: \begin{equation}\begin{aligned}\label{eqn:Efron1}
L_j^{Ef} = \prod_{i \in D_{(j)}} \frac{1}{\sum_{k\in R_j}\text{exp}(\Delta_{k,i})-\frac{h_i -1}{d_{(j)}}\sum_{m\in D_{(j)}}\text{exp}(\Delta_{m,i})}
\end{aligned}\end{equation}

where for the \(i\)th subject such that \(i\in D_{(j)}\), its partial
likelihood will be: \begin{equation}\begin{aligned}\label{eqn:Efron2}
L_{j_{(i)}}^{Ef} =\frac{1}{\sum_{k\in R_j}\text{exp}(\Delta_{k,i})-\frac{h_i -1}{d_{(j)}}\sum_{m\in D_{(j)}}\text{exp}(\Delta_{m,i})}
\end{aligned}\end{equation} and its logarithm will be:
\begin{equation}\begin{aligned}\label{eqn:Efron3}
l_{j_{(i)}}^{Ef} = -\text{log}\bigg \{1+ \sum_{k\in R_j, k\neq i}\text{exp}(\Delta_{1,i}-\Delta_{1,k})- \frac{h_i-1}{d_{(j)}}\big[\sum_{m\in D_{(j)}, m\neq i}\text{exp}(\Delta_{1,i}-\Delta_{1,m})\big] \bigg\}
\end{aligned}\end{equation}

From this log-likelihood, we can derive that:
\begin{equation}\begin{aligned}\label{eqn:Efron4}
\frac{\partial l_{j_{(i)}}^{Ef}}{\partial \Delta_{1,i}} = -\frac{\sum_{k\in R_j, k\neq i}\text{exp}(\Delta_{1,i}-\Delta_{1,k})-\frac{h_i-1}{d_{(j)}}\big[\sum_{m\in D_{(j)}, m\neq i}\text{exp}(\Delta_{1,i}-\Delta_{1,m})\big]}{\bigg\{1+\sum_{k\in R_j, k\neq i}\text{exp}(\Delta_{1,i}-\Delta_{1,k})-\frac{h_i-1}{d_{(j)}}\big[1+\sum_{m\in D_{(j)}, m\neq i}\text{exp}(\Delta_{1,i}-\Delta_{1,m})\big]\bigg\}}
\end{aligned}\end{equation}

If \(w \notin R_{(j)}\), then the derivative just simplifies to:
\begin{equation}\begin{aligned}\label{eqn:Efron6}
\frac{\partial l_{j_{(i)}}^{Ef}}{\partial \Delta_{1,w}} = 0
\end{aligned}\end{equation}

Therefore, assumes \(w \in D_{(j)}\), then we have:
\begin{equation}\begin{aligned}\label{eqn:Efron7}
\frac{\partial l_{j_{(i)}}^{Ef}}{\partial \Delta_{1,w}} = -\frac{-\text{exp}(\Delta_{1,i}-\Delta_{1,w})+\frac{h_i-1}{d_{(j)}}\text{exp}(\Delta_{1,i}-\Delta_{1,w})}{\bigg\{1+\sum_{k\in R_j, k\neq i}\text{exp}(\Delta_{1,i}-\Delta_{1,k})-\frac{h_i-1}{d_{(j)}}\big[1+\sum_{m\in D_{(j)}, m\neq i}\text{exp}(\Delta_{1,i}-\Delta_{1,m})\big]\bigg\}}
\end{aligned}\end{equation}

Define the notation \(\Box\) as below:
\begin{equation}\begin{aligned}\label{eqn:DefEfron1}
\Box = -\text{exp}(\Delta_{1,i}-\Delta_{1,w})+\frac{h_i-1}{d_{(j)}}\text{exp}(\Delta_{1,i}-\Delta_{1,w})
\end{aligned}\end{equation}

Therefore, we know that:
\begin{equation}\begin{aligned}\label{eqn:DefEfron1know}
\frac{\partial\Box}{\partial\Delta_{1,w}}=\text{exp}(\Delta_{1,i}-\Delta_{1,w})-\frac{h_i-1}{d_{(j)}}\text{exp}(\Delta_{1,i}-\Delta_{1,w}) = \Box
\end{aligned}\end{equation} and
\begin{equation}\begin{aligned}\label{eqn:DefEfron1know2}
\frac{\partial\Box}{\partial\Delta_{1,i}}=\Box
\end{aligned}\end{equation}

Define the notation \(\lozenge\) as below:
\begin{equation}\begin{aligned}\label{eqn:DefEfron2}
\lozenge = \sum_{k\in R_j, k\neq i}\text{exp}(\Delta_{1,i}-\Delta_{1,k})-\frac{h_i-1}{d_{(j)}}\big[\sum_{m\in D_{(j)}, m\neq i}\text{exp}(\Delta_{1,i}-\Delta_{1,m})\big]
\end{aligned}\end{equation} It can be shown that
\(\frac{\partial\lozenge}{\partial \Delta_{1,w}}=\Box\),
\(\frac{\partial\lozenge}{\partial \Delta_{1,i}}=\lozenge\) and
similarly that: \begin{equation}\begin{aligned}\label{eqn:DefEfron2back}
\frac{\partial l_{j_{(i)}}^{Ef}}{\partial\Delta_{1,w}} = -\frac{\Box}{\lozenge+\frac{d_{(j)}-h_i+1}{d_{(j)}}} \\
\frac{\partial l_{j_{(i)}}^{Ef}}{\partial\Delta_{1,i}} = -\frac{\lozenge}{\lozenge+\frac{d_{(j)}-h_i+1}{d_{(j)}}} \\
\end{aligned}\end{equation}

Consequently, the second derivatives of log-likelihood can be shown to
be:

\begin{equation}\begin{aligned}\label{eqn:Efron8}
\frac{\partial^2 l_{j_{(i)}}^{Ef}}{\partial \Delta_{1,w} \partial \Delta_{1,i}} = -\frac{\Box[\frac{d_{(j)}-h_i+1}{d_{(j)}}]}{\bigg[\lozenge+\frac{d_{(j)}-h_i+1}{d_{(j)}}\bigg]^2}
\end{aligned}\end{equation}

\begin{equation}\begin{aligned}\label{eqn:Efron9}
\frac{\partial^2 l_{j_{(i)}}^{Ef}}{\partial \Delta_{1,w}^2} &= -\frac{\frac{\partial\Box}{\partial\Delta_{1,w}}[\lozenge+\frac{d_{(j)}-h_i+1}{d_{(j)}}]-\Box^2}{\bigg[\lozenge+\frac{d_{(j)}-h_i+1}{d_{(j)}}\bigg]^2} \\
\\
&= \frac{\Box \big[\lozenge + \frac{d_{(j)}-h_i+1}{d_{(j)}}\big]+\Box^2}{\bigg[\lozenge+\frac{d_{(j)}-h_i+1}{d_{(j)}}\bigg]^2}
\end{aligned}\end{equation}

For \(M \in D_{(i)}\) and \(M \neq i,k\), we have:
\begin{equation}\begin{aligned}\label{eqn:Efron10}
\frac{\partial^2 l_{j_{(i)}}^{Ef}}{\partial \Delta_{1,w} \partial \Delta_{1,M}} = \frac{\Box\bigg[-\text{exp}(\Delta_{1,i}-\Delta_{1,M})+\frac{h_i-1}{d_{(j)}}\text{exp}(\Delta_{1,i}-\Delta_{1,M})\bigg]}{\bigg[\lozenge+\frac{d_{(j)}-h_i+1}{d_{(j)}}\bigg]^2}
\end{aligned}\end{equation}

Finally, for \(M \in R_j\) but \(M \notin D_j\), we have:
\begin{equation}\begin{aligned}\label{eqn:Efron11}
\frac{\partial l_{j_{(i)}}^{Ef}}{\partial \Delta_{1,M}} = -\frac{\text{exp}(\Delta_{1,i}-\Delta_{1,M})}{\lozenge + \frac{d_j-h_i+1}{d_j}}
\end{aligned}\end{equation}

\begin{equation}\begin{aligned}\label{eqn:Efron12}
\frac{\partial^2 l_{j_{(i)}}^{Ef}}{\partial \Delta_{1,M}^2} = \frac{-\text{exp}(\Delta_{1,i}-\Delta_{1,M})[\lozenge+\frac{d_j-h_i+1}{d_j}]+\text{exp}(2\Delta_{1,i}-2\Delta_{1,M})}{\bigg[\lozenge + \frac{d_j-h_i+1}{d_j}\bigg]^2}
\end{aligned}\end{equation}

\begin{equation}\begin{aligned}\label{eqn:Efron13}
\frac{\partial^2 l_{j_{(i)}}^{Ef}}{\partial \Delta_{1,M}\partial \Delta_{1,i}} = \frac{\text{exp}(\Delta_{1,i}-\Delta_{1,M})\frac{d_j-h_i+1}{d_j}}{\bigg[\lozenge + \frac{d_j-h_i+1}{d_j}\bigg]^2}
\end{aligned}\end{equation}

\begin{equation}\begin{aligned}\label{eqn:Efron14}
\frac{\partial^2 l_{j_{(i)}}^{Ef}}{\partial \Delta_{1,M}\partial \Delta_{1,w}} = \frac{-\text{exp}(\Delta_{1,i}-\Delta_{1,M})\Box}{\bigg[\lozenge + \frac{d_j-h_i+1}{d_j}\bigg]^2}
\end{aligned}\end{equation}

Now, assumes that \(a\in R_j\) but \(a \notin D_j\), then:
\begin{equation}\begin{aligned}\label{eqn:Efron15}
\frac{\partial^2 l_{j_{(i)}}^{Ef}}{\partial \Delta_{1,M}\partial \Delta_{1,a}} = \frac{\text{exp}(2\Delta_{1,i}-\Delta_{1,M}-\Delta_{1,a})\Box}{\bigg[\lozenge + \frac{d_j-h_i+1}{d_j}\bigg]^2}
\end{aligned}\end{equation}

Using these information above, we can construct the negated Hessian
matrix \(C_{j_{(i)}}\) for this \(i\)th subject, and therefore construct
the negated Hessian matrix \(C_j\) of the log-likelihood at time
\(t_{(j)}\), which is \(C_j = \sum_{i\in D_{(j)}}C_{j_{(i)}}\).
Consequently, we are able to get the C matrix of the full sample by
summing these \(C_j\) matrixes at different observed times.

\hypertarget{approximation-using-full-likelihood-with-left-truncation}{%
\subsection{Approximation using full-likelihood with
left-truncation:}\label{approximation-using-full-likelihood-with-left-truncation}}

If there are both right-censoring and left-truncations in our data-set,
the ``data augmentation trick'' that INLA uses actually still works
theoretically, but the software does not actually allow the user to run
the approximation under this scenario. Fortunately, it can be solved
using the package of the new proposed algorithm.

Recall that left-truncation happens when we cannot observed the i-th
individual lifetime \(t_i\), unless it is greater than the entry time
\(u_i\). Under this setup, all the observed lifetimes \(t_i\)'s are
known to be greater than their corresponding entry times \(u_i\)'s. In
other words, we should use conditional probability given \(t_i > u_i\)
to form our likelihood. For simplicity, let's still consider the same
semi-parametric proportional hazard model with piece-wise constant
basline hazard.

Denote the i-th lifetime as \(t_i\), the i-th left truncation time is
\(u_i\), and assume that \(t_i \in (s_{k_{(i)}-1},s_{k_{(i)}}]\),
\(u_i \in (s_{m_{(i)}-1},s_{m_{(i)}}]\). Therefore, we have the
likelihood being:

\begin{equation}\begin{aligned}\label{eqn:lt}
L &= \prod_{i=1}^{n} {\bigg[\frac{f(t_i)}{S(u_i)}\bigg]^{\delta_i}\bigg[\frac{S(t_i)}{S(u_i)}\bigg]^{1- \delta_i}} \\
  &= \prod_{i=1}^{n} {f(t_i)^{\delta_i}}{\frac{S(t_i)^{1-\delta_i}}{S(u_i)}} \\
  &= \prod_{i=1}^{n} {h(t_i)^{\delta_i}}{\frac{S(t_i)}{S(u_i)}} \\
\end{aligned}\end{equation}

Using the likelihood above, we can easily derive the log-likelihood
being:

\begin{equation}\begin{aligned}\label{eqn:lt_log}
l &= \sum_{i=1}^{n} {{\delta_i}\text{log}[h(t_i)]} + \sum_{i=1}^{n} {\big[\text{log}S(t_i)-\text{log}S(u_i)\big]}  \\
\end{aligned}\end{equation}

Recall that if \(t_i \in (s_{k_{(i)}-1},s_{k_{(i)}}]\), and
\(u_i \in (s_{m_{(i)}-1},s_{m_{(i)}}]\), we have the followings:
\begin{equation}\begin{aligned}\label{eqn:ltcase1}
\text{log}S(t_i) &= -\int_{0}^{t_i} h(x) dx \\
                 &= -\sum_{j=1}^{k_{(i)}-1} (S_j-S_{j-1})\text{exp}(\eta_{ij})-(t_i-S_{k_{(i)}-1})\text{exp}(\eta_{ik_{(i)}}) \\
\end{aligned}\end{equation}

Similarly:

\begin{equation}\begin{aligned}\label{eqn:ltcase2}
\text{log}S(u_i) &= -\int_{0}^{u_i} h(x) dx \\
                 &= -\sum_{j=1}^{m_{(i)}-1} (S_j-S_{j-1})\text{exp}(\eta_{ij})-(u_i-S_{m_{(i)}-1})\text{exp}(\eta_{im_{(i)}}) \\
\end{aligned}\end{equation}

Therefore, the difference between this two terms can be written as:
\begin{equation}\begin{aligned}\label{eqn:diff}
\text{log}S(t_i)-\text{log}S(u_i) = -\int_{u_i}^{t_i} h(x) dx \\
\end{aligned}\end{equation} Whereas:
\begin{equation}\begin{aligned}\label{eqn:diff2}
-\int_{u_i}^{t_i} h(x) dx = -\sum_{j=m_{(i)}+1}^{k_{(i)}-1} (S_j-S_{j-1})\text{exp}(\eta_{ij}) - (S_{m_{(i)}}-u_i)\text{exp}(\eta_{im_{(i)}})-(t_i - S_{k_{(i)}-1})\text{exp}(\eta_{n_{ik_{(i)}}})
\end{aligned}\end{equation}

Then, combine all of the information above together, we can derive an
expression for the log-likelihood of the sample:
\begin{equation}\begin{aligned}\label{eqn:all_together_logoftr}
l = \sum_{i=1}^{n} {\delta_i}{\eta_{i{k_{(i)}}}}-\sum_{i=1}^{n}\sum_{j=m_{(i)}+1}^{k_{(i)}-1}(S_j-S_{j-1})\text{exp}(\eta_{ij}) - \sum_{i=1}^{n} (S_{m_{(i)}}-u_i)\text{exp}(\eta_{im_{(i)}}) - \sum_{i=1}^{n} (t_i-S_{k_{(i)}-1})(\eta_{ik_{(i)}})
\end{aligned}\end{equation}

If we have \(m_{(i)} \leq k_{(i)}-1\), then the above expression
simplify to:
\begin{equation}\begin{aligned}\label{eqn:all_together_logoftr2}
l = \sum_{i=1}^{n} {\delta_i}{\eta_{i{k_{(i)}}}}- \sum_{i=1}^{n} (S_{m_{(i)}}-u_i)\text{exp}(\eta_{im_{(i)}}) - \sum_{i=1}^{n} (t_i-S_{k_{(i)}-1})(\eta_{ik_{(i)}})
\end{aligned}\end{equation}

Next step, I will derive the corresponding C and Q matrix in this case.

\hypertarget{derivation-of-c-matrix-with-left-truncation}{%
\subsubsection{Derivation of C-matrix with
left-truncation}\label{derivation-of-c-matrix-with-left-truncation}}

To make the derivation most general, I will assume that
\(k_{(i)}-1 \geq m_{(i)}+1\), since otherwise the computation will be
simplified to trivial. For the i-th observation \(t_{(i)}\) with
left-truncation time \(u_{(i)}\), assume that:
\(t_i \in (s_{k_{(i)}-1},s_{k_{(i)}}]\), and
\(u_i \in (s_{m_{(i)}-1},s_{m_{(i)}}]\), then the likelihood of this
observation will be:
\begin{equation}\begin{aligned}\label{eqn:ind_i_log}
l = \delta_i \eta_{ik_{(i)}} - \sum_{m_{(i)}+1}^{k_{(i)}-1} (S_j - S_{j-1}) \text{exp}(\eta_{ij}) - (S_{m_{(i)}}-u_{(i)})\text{exp}(\eta_{im_{(i)}})-(t_{(i)}-S_{k_{(i)}-1})\text{exp}(\eta_{ik_{(i)}})
\end{aligned}\end{equation}

For \(j<m_{(i)}\) or \(j>k_{(i)}\), apparently we have
\(\frac{ \partial l}{\partial \eta_{ij}}\) =
\(\frac{ \partial^2 l}{\partial \eta_{ij}^2}\) = 0.

For \(j = m_{(i)}\), we can compute that
\(\frac{ \partial l}{\partial \eta_{ij}}\) =
\(\frac{ \partial^2 l}{\partial \eta_{ij}^2}\) =
\(- (S_{m_{(i)}}-u_{(i)})\text{exp}(\eta_{im_{(i)}})\).

For \(m_{(i)} < j < k_{(i)}\),
\(\frac{ \partial l}{\partial \eta_{ij}}\) =
\(\frac{ \partial^2 l}{\partial \eta_{ij}^2}\) =
\(- (S_j-S_{j-1})\text{exp}(\eta_{ij})\).

For \(j = k_{(i)}\), it can be shown that:

\begin{equation}\begin{aligned}\label{eqn:smalladd}
\frac{ \partial l}{\partial \eta_{ij}} = \delta_i - (t_{(i)}-S_{k_{(i)}-1})\text{exp}(\eta_{ik_{(i)}}) \\
\end{aligned}\end{equation} So,
\begin{equation}\begin{aligned}\label{eqn:smalladd2}
\frac{ \partial^2 l}{\partial \eta_{ij}^2} = - (t_{(i)}-S_{k_{(i)}-1})\text{exp}(\eta_{ik_{(i)}})
\end{aligned}\end{equation}

From now, let's denote \(\text{exp}(\eta_{ij})\) as \(b_{ij}\). Now, we
can use the above information, to obtain the C-matrix (negated Hessian)
of the i-th observation:

\begin{equation}
\begin{pmatrix} 
0 &   &  & \cdots & \cdots &  & \cdots \\
0 & \ddots  &   &  & && &  \\
\vdots & \cdots & (S_{m_{(i)}}-u_{(i)}) b_{im_{(i)}} &  & \cdots & \cdots & \cdots &  \\
\vdots & & &(S_{m_{(i)}+1}-S_{m_{(i)}})b_{i(m_{(i)}+1)} &  & \cdots & \cdots& \\
\vdots & & & \ddots  & \ddots & \ddots & \cdots& \\
\vdots  & & & \cdots & (S_{k_{(i)}-1}-S_{k_{(i)}-2})b_{i(k_{(i)}-1)}  & \cdots &  \\
\vdots  & && & \cdots & (t_i-S_{k_{(i)}-1})b_{ik_{(i)}} & \cdots\\
0 & \cdots &  \cdots & \cdots & \cdots & \cdots & \ddots \\
\end{pmatrix}
\end{equation} This is a matrix with \(k \times k\) dimension.

Let's call the C-matrix of observation i as \(C_i\), then the C-matrix
of the whole sample will be: \begin{equation}
C = \begin{pmatrix} 
C_1 & 0 & 0 & \cdots & & \\ 
0 & C_2 & 0 & \cdots & & \\
  & \cdots & \ddots &  & & \\
& & & & C_n & \cdots & \vdots \\ 
& & & \ddots & &&\vdots \\
& & & & & & 0
\end{pmatrix}
\end{equation} This C matrix has a dimension of
\((nk+\sum_{q=1}^{R}M_q + p +k)\) \(\times\)
\((nk+\sum_{q=1}^{R}M_q + p +k)\), and each block matrix \(C_i\) has the
same dimension being \(k\times k\).

We can see that the present of left-truncation does not change the
overall shape of the C-matrix. The only effect of it is to change the
diagonal terms of each individual observation's \(C_i\) matrix depending
on the i-th left-truncation time. Therefore, the computation efficiency
and precision will not be affected too much.

\newpage

\hypertarget{approximation-using-full-likelihood-with-interval-censoring}{%
\subsection{Approximation using full-likelihood with
interval-censoring:}\label{approximation-using-full-likelihood-with-interval-censoring}}

Suppose that we are not observing the exact lifetimes of individiuals,
but only the set of intervals that contain each lifetime. In other
words, our data-set is \(\{L_i, R_i ; i=1,...n\}\) , where
\(L_i \leq t_i \leq R_i\). Using the same way to define the piece-wise
constant hazard functions as before, we can assume that for the i-th
observation, we have \(R_i\in(S_{k_{(i)}-1},S_{k_{(i)}}]\), and
\(L_i\in(S_{m_{(i)}-1},S_{m_{(i)}}]\).

Now, we can write down the log-likelihood of the i-th individual using
the above information:

\begin{equation}\begin{aligned}\label{eqn:icl1}
l_i &= \delta_i \text{log}\big[h_i(R_i)\big] - \int_{0}^{R_i} h_i(u) du + \text{log}\bigg\{ 1-\text{exp}\big[ -\int_{L_i}^{R_i} h_i(u)du \big]  \bigg\} \\
    &= \delta_i \eta_{ik_{i}} - \sum_{j=1}^{k_{(i)}-1} \text{exp}(\eta_{ij}) - (R_i-S_{k_{(i)}-1}) \text{exp}(\eta_{ik_{(i)}}) + \text{log}\big[1-\text{exp}(\vartheta_i)\big] \\
\end{aligned}\end{equation}

Where \(\vartheta_i\) is defined as:
\begin{equation}\begin{aligned}\label{eqn:vartheta}
\vartheta_i = -\int_{L_i}^{R_i} h_i(u)du = -\sum_{j=m_{(i)}+1}^{k_{(i)}-1} (S_j - S_{j-1})\text{exp} (\eta_{ij}) -(S_{m_{(i)}} - L_i)\text{exp}(\eta_{im_{(i)}})-(R_i - S_{k_{(i)}-1})\text{exp}(\eta_{ik_{(i)}})
\end{aligned}\end{equation}

Now, we can take derivative of this log-likelihood with respect to the
ij-th linear predictor (assume that \(m_{(i)}+1\leq j \leq k_{(i)}-1\)),
and get the following result:

\begin{equation}\begin{aligned}\label{eqn:ijderivofsingle}
\frac{\partial l_i}{\partial \eta_{ij}} = -(S_j-S_{j-1})\text{exp}(\eta_{ij}) -  \frac{\text{exp}(\vartheta_i) \frac{\partial \vartheta_i}{\partial \eta_{ij}}}{1-\text{exp}(\vartheta_i)}
\end{aligned}\end{equation}

Where \(\frac{\partial \vartheta_i}{\partial\eta_{ij}}\) is
\(-(S_j-S_{j-1})\text{exp}(\eta_{ij})\) in this case.

Since \(\vartheta_i\) depends on more than one linear predictors, so it
follows naturally that \(\frac{\partial l_i}{\partial \eta_{ij}}\) will
be a function of several different linear predictors, which means the
C-matrix of each observation's log-likelihood will be non-diagonal. That
will be a serious problem for INLA, as its package relies on the
diagonality of C-matrix, but it will not cause any probelm for our new
proposed algorithm, as the diagonality of C-matrix is no longer required
here.

\hypertarget{example-from-diabetics-data-set}{%
\section{Example from Diabetics
Data-set}\label{example-from-diabetics-data-set}}

Firstly, I will use the data-set ``diabetics'' to demonstrate the
equivalence between ``coxph'' approach in survival package, INLA's
approach and our proposed approach. This data-set contains the results
from a trail of laser coagulation for the treatment of diabetic
retinopathy from 197 patients. Each patient had one eye randomized to
laser treatment and the other eye received no treatment. (To be
continued on ``proposed approach'')

The variable ``id'' specifies the subject's ID.

The variable ``laser'' is a categorical variable with levels xenon or
argon.

The variable ``age'' is the age of the subject at diagnosis.

The variable ``eye'' is a categorical variable with levels left or
right.

The variable ``trt'' is a categorical variable with levels 0 for no
treatment and 1 for treatment using laser.

The variable ``risk'' classifies the risk levels of the patients.

The response variable in this data-set will be ``time'', which are the
actual time to blindness in months, minus the minimum possible time to
event (6.5 months), and ``status'' indicates whether the time is
censored with 1 for visual loss and 0 for censored. The censoring can be
due to death, dropout, or end of the study.

Let use briefly view the structure of ``diabetics'':

\hypertarget{data-set}{%
\subsection{Data-set:}\label{data-set}}

\begin{Shaded}
\begin{Highlighting}[]
\KeywordTok{head}\NormalTok{(}\KeywordTok{as_tibble}\NormalTok{(diabetic))}
\end{Highlighting}
\end{Shaded}

\begin{verbatim}
## # A tibble: 6 x 8
##      id laser   age eye     trt  risk  time status
##   <int> <fct> <int> <fct> <int> <int> <dbl>  <int>
## 1     5 argon    28 left      0     9  46.2      0
## 2     5 argon    28 right     1     9  46.2      0
## 3    14 xenon    12 left      1     8  42.5      0
## 4    14 xenon    12 right     0     6  31.3      1
## 5    16 xenon     9 left      1    11  42.3      0
## 6    16 xenon     9 right     0    11  42.3      0
\end{verbatim}

We can see that those survival times are right-censored. We will fit a
cox proportional hazard model with piece-wise constant baseline hazard,
assuming that each individual will have the same baseline hazard
function. The variable ID will be treated as a random effect(which can
be added using code ``frailty.gaussian(id)''). The variables ``age'',
``eye'', ``trt'' and ``laser'' will be included as fixed effects.

\hypertarget{survival-coxph}{%
\subsection{Survival: coxph}\label{survival-coxph}}

\begin{Shaded}
\begin{Highlighting}[]
\NormalTok{diabetic.CoxPh <-}\StringTok{ }\KeywordTok{coxph}\NormalTok{(}\KeywordTok{Surv}\NormalTok{(time, status)}\OperatorTok{~}\NormalTok{age }\OperatorTok{+}\StringTok{ }\NormalTok{eye }\OperatorTok{+}\StringTok{ }\NormalTok{trt }\OperatorTok{+}\StringTok{ }\NormalTok{laser }\OperatorTok{+}\StringTok{ }\KeywordTok{frailty.gaussian}\NormalTok{(id), }\DataTypeTok{data =}\NormalTok{ diabetic)}
\KeywordTok{summary}\NormalTok{(diabetic.CoxPh)}
\end{Highlighting}
\end{Shaded}

\begin{verbatim}
## Call:
## coxph(formula = Surv(time, status) ~ age + eye + trt + laser + 
##     frailty.gaussian(id), data = diabetic)
## 
##   n= 394, number of events= 155 
## 
##                      coef      se(coef) se2      Chisq  DF    p      
## age                   0.009548 0.01323  0.009879   0.52  1.00 4.7e-01
## eyeright              0.483005 0.17501  0.168693   7.62  1.00 5.8e-03
## trt                  -1.007507 0.17930  0.174315  31.57  1.00 1.9e-08
## laserargon           -0.182388 0.39471  0.293566   0.21  1.00 6.4e-01
## frailty.gaussian(id)                             131.35 79.63 2.4e-04
## 
##            exp(coef) exp(-coef) lower .95 upper .95
## age           1.0096     0.9905    0.9837    1.0361
## eyeright      1.6209     0.6169    1.1503    2.2842
## trt           0.3651     2.7388    0.2569    0.5189
## laserargon    0.8333     1.2001    0.3844    1.8062
## 
## Iterations: 6 outer, 24 Newton-Raphson
##      Variance of random effect= 0.9447455 
## Degrees of freedom for terms=  0.6  0.9  0.9  0.6 79.6 
## Concordance= 0.867  (se = 0.867 )
## Likelihood ratio test= 228  on 82.62 df,   p=2e-15
\end{verbatim}

From the output above, it can be seen that variables ``age'' and
``eyeright'' have positive association with the rate of occurrence of
visual loss, and variables ``trt'' and whether using ``argon'' type of
laser are negatively associated with the rate. All the fixed effects
that we included in this study have significant effects for the risk of
visual loss.

\hypertarget{bayesian-inla}{%
\subsection{Bayesian: INLA}\label{bayesian-inla}}

Now we fit the same model using INLA:

\begin{Shaded}
\begin{Highlighting}[]
\NormalTok{formula =}\StringTok{ }\KeywordTok{inla.surv}\NormalTok{(time, status) }\OperatorTok{~}\StringTok{ }\NormalTok{age }\OperatorTok{+}\StringTok{ }\NormalTok{eye }\OperatorTok{+}\StringTok{ }\NormalTok{trt }\OperatorTok{+}\StringTok{ }\NormalTok{laser }\OperatorTok{+}\StringTok{ }\KeywordTok{f}\NormalTok{(id, }\DataTypeTok{model =} \StringTok{"iid"}\NormalTok{)}
\NormalTok{diabetic.INLA <-}\StringTok{ }\KeywordTok{inla}\NormalTok{(formula, }\DataTypeTok{control.compute =} \KeywordTok{list}\NormalTok{(}\DataTypeTok{dic =} \OtherTok{TRUE}\NormalTok{), }\DataTypeTok{family =} \StringTok{"coxph"}\NormalTok{, }
                      \DataTypeTok{data =}\NormalTok{ diabetic, }\DataTypeTok{control.hazard=}\KeywordTok{list}\NormalTok{(}\DataTypeTok{model=}\StringTok{"rw2"}\NormalTok{, }\DataTypeTok{n.intervals=}\DecValTok{20}\NormalTok{))}
\NormalTok{diabetic.INLA}\OperatorTok{$}\NormalTok{summary.fixed}
\end{Highlighting}
\end{Shaded}

\begin{verbatim}
##                     mean         sd  0.025quant     0.5quant  0.975quant
## (Intercept) -4.763672659 0.24991497 -5.27233955 -4.757647439 -4.28886760
## age          0.007158258 0.01085298 -0.01385058  0.007023359  0.02906263
## eyeright     0.385650097 0.17633405  0.04932528  0.382111768  0.74263814
## trt         -0.867087249 0.19233459 -1.26403168 -0.860221093 -0.50842431
## laserargon  -0.118785427 0.32286308 -0.77627597 -0.111805767  0.49849997
##                    mode          kld
## (Intercept) -4.74602026 3.792471e-06
## age          0.00683526 8.317970e-06
## eyeright     0.37517726 6.138726e-06
## trt         -0.84600084 1.472575e-05
## laserargon  -0.09905692 3.891902e-06
\end{verbatim}

It seems like these two approaches are similar enough. Though in the
classic ``coxph'' approach, the effects of age and using argon-type
laser are significant, but INLA gives insignificant results(the 95\%
credible interval contains zero). There is an estimate for intercept in
the INLA's method because we used a random walk prior in that.

\hypertarget{example-from-bladder-data-set}{%
\section{Example from Bladder
Data-set:}\label{example-from-bladder-data-set}}

Next, we will study the two approaches on the data-set ``bladder1''.
This is the full data-set that contains the result from a study on
recurrences of bladder cancer from 118 subjects. In this data-set, the
variables that we are interested in are ``id'', ``number'', ``size'',
``recur'', ``times'' and ``censored''.

The variable ``id'' is just the patient ID.

The variable ``number'' specifies initial number of tumors of each
subject.

The variable ``size'' is the size of largest initial tumor.

The variable ``recur'' is the number of recurrence of bladder cancer for
that subject.

The response variable will be ``time'' which is computed to be the
duration of times until recurrence or death, censored by the variable
``censored'' with 0 means being censored.

\hypertarget{data-set-1}{%
\subsection{Data-set:}\label{data-set-1}}

\begin{Shaded}
\begin{Highlighting}[]
\NormalTok{data <-}\StringTok{ }\KeywordTok{as_tibble}\NormalTok{(bladder1)}
\NormalTok{data <-}\StringTok{ }\KeywordTok{select}\NormalTok{(data,}\OperatorTok{-}\KeywordTok{c}\NormalTok{(rsize,rtumor,enum))}
\NormalTok{data <-}\StringTok{ }\NormalTok{data }\OperatorTok\StringTok{ }\KeywordTok{mutate}\NormalTok{(}\DataTypeTok{censored =}\NormalTok{ status}\OperatorTok{==}\DecValTok{0}\NormalTok{)}
\ControlFlowTok{for}\NormalTok{ (i }\ControlFlowTok{in} \DecValTok{1}\OperatorTok{:}\KeywordTok{length}\NormalTok{(data}\OperatorTok{$}\NormalTok{censored)) \{}
  \ControlFlowTok{if}\NormalTok{(data}\OperatorTok{$}\NormalTok{censored[i]) data}\OperatorTok{$}\NormalTok{censored[i] <-}\StringTok{ }\DecValTok{0}
  \ControlFlowTok{else}\NormalTok{ data}\OperatorTok{$}\NormalTok{censored[i] <-}\StringTok{ }\DecValTok{1}
\NormalTok{\}}
\NormalTok{data <-}\StringTok{ }\NormalTok{data }\OperatorTok\StringTok{ }\KeywordTok{mutate}\NormalTok{(}\DataTypeTok{times =}\NormalTok{ stop}\OperatorTok{-}\NormalTok{start)}
\KeywordTok{head}\NormalTok{(data)}
\end{Highlighting}
\end{Shaded}

\begin{verbatim}
## # A tibble: 6 x 10
##      id treatment number  size recur start  stop status censored times
##   <int> <fct>      <int> <int> <int> <int> <int>  <dbl>    <dbl> <int>
## 1     1 placebo        1     1     0     0     0      3        1     0
## 2     2 placebo        1     3     0     0     1      3        1     1
## 3     3 placebo        2     1     0     0     4      0        0     4
## 4     4 placebo        1     1     0     0     7      0        0     7
## 5     5 placebo        5     1     0     0    10      3        1    10
## 6     6 placebo        4     1     1     0     6      1        1     6
\end{verbatim}

\begin{Shaded}
\begin{Highlighting}[]
\KeywordTok{tail}\NormalTok{(data)}
\end{Highlighting}
\end{Shaded}

\begin{verbatim}
## # A tibble: 6 x 10
##      id treatment number  size recur start  stop status censored times
##   <int> <fct>      <int> <int> <int> <int> <int>  <dbl>    <dbl> <int>
## 1   115 thiotepa       4     1     3    24    47      1        1    23
## 2   115 thiotepa       4     1     3    47    50      0        0     3
## 3   116 thiotepa       3     4     0     0    54      0        0    54
## 4   117 thiotepa       2     1     1     0    38      1        1    38
## 5   117 thiotepa       2     1     1    38    54      0        0    16
## 6   118 thiotepa       1     3     0     0    59      3        1    59
\end{verbatim}

Here the variable ``id'' specifies different individuals, and should be
treated as a random effect. The variable ``time'' is computed using the
difference between variables ``start'' and ``stop'', which denote the
start time and end time of each time interval. It seems like a interval
censoring problem but the start time is known before hand, so we can
treat it as a regular type-I right censoring.

In this study, a nonzero value of ``status'' can be death from bladder
disease, death from other reason or recurrence. Here we will just view
all of these situations as ``occurrence'' for simplicity. So the
variable ``censored'' is created such that it is 1 if ``status'' is
non-zero, otherwise 0. We will include ``number'', ``size'' and
``recur'' as fixed effects in this study.

\hypertarget{survivalcoxph}{%
\subsection{Survival:coxph}\label{survivalcoxph}}

\begin{Shaded}
\begin{Highlighting}[]
\NormalTok{bladder.CoxPh <-}\StringTok{ }\KeywordTok{coxph}\NormalTok{(}\KeywordTok{Surv}\NormalTok{(times, censored)}\OperatorTok{~}\StringTok{ }\NormalTok{number }\OperatorTok{+}\StringTok{ }\NormalTok{size }\OperatorTok{+}\StringTok{ }\NormalTok{recur }\OperatorTok{+}\StringTok{ }\KeywordTok{frailty.gaussian}\NormalTok{(id), }\DataTypeTok{data =}\NormalTok{ data)}
\KeywordTok{summary}\NormalTok{(bladder.CoxPh)}
\end{Highlighting}
\end{Shaded}

\begin{verbatim}
## Call:
## coxph(formula = Surv(times, censored) ~ number + size + recur + 
##     frailty.gaussian(id), data = data)
## 
##   n= 294, number of events= 218 
## 
##                      coef     se(coef) se2     Chisq DF   p      
## number               0.064848 0.04246  0.04051  2.33 1.00 1.3e-01
## size                 0.008336 0.04797  0.04631  0.03 1.00 8.6e-01
## recur                0.229848 0.02559  0.02431 80.66 1.00 2.7e-19
## frailty.gaussian(id)                            5.12 4.59 3.5e-01
## 
##        exp(coef) exp(-coef) lower .95 upper .95
## number     1.067     0.9372    0.9818     1.160
## size       1.008     0.9917    0.9179     1.108
## recur      1.258     0.7947    1.1968     1.323
## 
## Iterations: 8 outer, 43 Newton-Raphson
##      Variance of random effect= 0.0257231 
## Degrees of freedom for terms= 0.9 0.9 0.9 4.6 
## Concordance= 0.694  (se = 0.694 )
## Likelihood ratio test= 105.4  on 7.33 df,   p=<2e-16
\end{verbatim}

Fitting this model using the traditional partial likelihood approach
gives insignificant results for all the fixed effects except ``recur'',
which has a strong positive effect. But we will still proceed to check
what will happen if we fit it using a Bayesian approach.

\hypertarget{bayesian-inla-1}{%
\subsection{Bayesian: INLA}\label{bayesian-inla-1}}

\begin{Shaded}
\begin{Highlighting}[]
\NormalTok{formula =}\StringTok{ }\KeywordTok{inla.surv}\NormalTok{(times, censored) }\OperatorTok{~}\StringTok{ }\NormalTok{number }\OperatorTok{+}\StringTok{ }\NormalTok{size }\OperatorTok{+}\StringTok{ }\NormalTok{recur }\OperatorTok{+}\StringTok{ }\KeywordTok{f}\NormalTok{(id, }\DataTypeTok{model =} \StringTok{"iid"}\NormalTok{)}
\NormalTok{bladder.INLA <-}\StringTok{ }\KeywordTok{inla}\NormalTok{(formula, }\DataTypeTok{control.compute =} \KeywordTok{list}\NormalTok{(}\DataTypeTok{dic =} \OtherTok{TRUE}\NormalTok{), }\DataTypeTok{family =} \StringTok{"coxph"}\NormalTok{, }
                      \DataTypeTok{data =}\NormalTok{ data, }\DataTypeTok{control.hazard=}\KeywordTok{list}\NormalTok{(}\DataTypeTok{model=}\StringTok{"rw2"}\NormalTok{, }\DataTypeTok{n.intervals=}\DecValTok{20}\NormalTok{))}
\NormalTok{bladder.INLA}\OperatorTok{$}\NormalTok{summary.fixed}
\end{Highlighting}
\end{Shaded}

\begin{verbatim}
##                     mean         sd  0.025quant    0.5quant 0.975quant
## (Intercept) -3.945622483 0.26869284 -4.49396505 -3.93815533 -3.4393035
## number       0.061328621 0.04045248 -0.02007874  0.06202239  0.1388195
## size         0.009253811 0.04631391 -0.08484239  0.01037921  0.0970400
## recur        0.215846324 0.02392149  0.16899581  0.21580588  0.2628817
##                    mode          kld
## (Intercept) -3.92341008 5.445426e-06
## number       0.06339939 2.828601e-07
## size         0.01261178 7.224525e-07
## recur        0.21572710 1.742875e-06
\end{verbatim}

Indeed, the two results seem pretty similar in general. In both cases,
we can see that there are no apparent relationships between all of the
fixed effects and the rate of occurrence of bladder cancer's recurrence,
or death, except the variable ``recur'' with a positive effect.


\end{document}
